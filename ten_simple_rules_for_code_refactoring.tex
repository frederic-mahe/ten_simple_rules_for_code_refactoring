\documentclass[10pt,letterpaper]{article}

\input{./plos-latex-template/plos2015.sty}

\begin{document}
\vspace*{0.2in}

% Title must be 250 characters or less.
\begin{flushleft}
  {\Large
    \textbf\newline{Ten simple rules for code refactoring}
  }
  \newline
  \\
  Name1 Surname\textsuperscript{1,2},
  Name2 Surname\textsuperscript{2},
  Name3 Surname\textsuperscript{2,3},
  Name4 Surname\textsuperscript{2},
  Name5 Surname\textsuperscript{2},
  Name6 Surname\textsuperscript{2},
  Name7 Surname\textsuperscript{1,2,3*}
  \\
  \bigskip
  \textbf{1} Affiliation Dept/Program/Center, Institution Name, City, State, Country
  \\
  \textbf{2} Affiliation Dept/Program/Center, Institution Name, City, State, Country
  \\
  \textbf{3} Affiliation Dept/Program/Center, Institution Name, City, State, Country
  \\
  \bigskip

  % corresponding author
  * frederic.mahe@cirad.fr

\end{flushleft}

\linenumbers

\section*{Introduction}

Why should we care about the quality of computational biology
software? and how can it be improved? The goal is to provide
guidelines on how test, refactor, strengthen and modernize existing
code (C/C++, python, R?). In short, how to improve code quality, as a
follow up on the softwipe initiative.

\section*{Rule 1: build a safeguard!}

Whether you are working on your own code, or being tasked to maintain
an existing code base, the first order of business should be to
document the code and to write unit tests. Your goal is to make sure
that the code does what it is supposed to do, on all the operating
systems it is supposed to do it on, and that it handles unexpected
situations gracefully.

Testing is useful for refactoring, preventing regressions, and
improving how code is written. Writing tests is also a good way to
become a better programmer and team member. Automatic tests leave you
with more time and energy to focus on software design. Tests can be
written in a language different than the tested code, for example
writting tests in bash for a compiled binary can significantly lower
the hurdle. Testing is a dark art, but here is a quickstarter on how
to write useful tests.

Tests are not meant to find bugs, but rather to prevent bugs from
coming back, to signal changes to the logic of your code, and to
better document your code. First, each fixed issue in your bug tracker
should be turned into the smallest possible test able to trigger the
bug. These are called regression tests and are often the only
available test suite. Second, unit tests should test all the important
input values (null, negative, odd, even, not a number, etc.) or
parameters accepted by your code. Integration tests should challenge
each assertion present in the code documentation, as well as the
complement or the oposite of each assertion. For instance, if the help
message says that the program accepts input values between 0 and 10,
does the program rejects values greater than 10? what about negative
values?  Keep in mind that exhaustive testing is not possible, but
smart testing can cover a lot of ground.

Now that you have automatic tests, you can start refactoring your
code.

\section*{Rule 2: measure code coverage}

find out what branches of your code your tests are missing. Find out
dead code.

\section*{Rule 3: fuzzy input}

Torture your code with unexpected input. Find corner cases,
crashes. Modify your code to handle weird cases gracefully. Update
your test base. Just-in-time documentation. You can't completely
preempt the creativity of end-users.

\section*{Rule 4: compilers/interpreters are your friends}

Use all the possibilities offered by your compilers or execution
environments. Use all available warnings (and fix them). Try different
compilers, and different versions of each compilers (gcc and llvm at
least) or different interpreters (python, pypy, etc.).

Compilers are getting smarter and pickier, and your code can benefit
from that.

Python ducumentation: Test your application with the '-W default'
command-line option to see DeprecationWarning and
PendingDeprecationWarning, or even with '-W error' to treat them as
errors. Warnings Filter can be used to ignore warnings from
third-party code.

\section*{Rule 5: memory managment and concurency are hard}

multi-core CPUS are the norm now. valgrind and helgrind. Find memory
leaks, race conditions. Add new tests to your test suite.

\section*{Rule 6: use analyzers}

cppcheck, clang-tidy, flake8. Analyzers can automatically detect large
classes of problems.

clang-tidy can modernize for you without modifying the underlying
logic (AST).

\section*{Rule 7: refactor}

The goal of refactoring is to change the underlying code but to
preserve the software's behavior. To refactor is not to introduce new
features or functionality changes. Refactoring should be performed
regularly, otherwise the technical debt will accumulate.

Automatic tests make it safer (safety net). Refactoring rules are:
- less moving parts (simpler code),
- use the STL or standard modules,
- simplify the design (fight feature creep, remove unused),
- move low-level actions to functions (high-level code should read
  like English),
- small incremental steps, rather than big leaps (tests should pass
  all the time),

Perfection is not realistic, focus on the most important parts.

Java and C\# have great factoring tools, as refactoring is culturally
more important in the industry. Easy with Java, use Reeks for Ruby.

\section*{Rule 8: profile to focus your refactoring efforts}

refactoring can yield performance improvements or reduced memory
footprint (or harm it if you are not careful). It is important to
benchmark the performances of your tool to avoid introducing
slowdowns. It happened to me while updating swarm. I initialized a
local variable in a loop, which doubled the runtime!

Use valgrind to count the number of instructions.

\section*{Rule 9: incremental improvement}

do not aim for perfection.

\section*{Rule 10: stay alert}

Avoid code decay. Languages evolve, introducing new features and
removing some. Keep your finger on the pulse, and check whether your
code can or should be modernized by adopting new features or by
replacing deprecated functions (you might want to maintain
compatibility with older versions such as python 3.4 or C+11).

For instance, check https://docs.python.org/3.9/whatsnew/3.9.html

\section*{Conclusion}

% Should we plot a summary workflow?

Towards an ethics charter for developpers

reduce the technical debt, reduce waste of hardware, time and energy
by making well tested, reliable and efficient software.

\section*{Acknowledgments}

Cras egestas velit mauris, eu mollis turpis pellentesque sit
amet. Interdum et malesuada fames ac ante ipsum primis in
faucibus. Nam id pretium nisi. Sed ac quam id nisi malesuada
congue. Sed interdum aliquet augue, at pellentesque quam rhoncus
vitae.

\nolinenumbers

% Either type in your references using
% \begin{thebibliography}{}
% \bibitem{}
% Text
% \end{thebibliography}
%
% or
%
% Compile your BiBTeX database using our plos2015.bst
% style file and paste the contents of your .bbl file
% here. See http://journals.plos.org/plosone/s/latex for
% step-by-step instructions.
%
\begin{thebibliography}{10}

\bibitem{bib1}
Conant GC, Wolfe KH.
\newblock {{T}urning a hobby into a job: how duplicated genes find new
  functions}.
\newblock Nat Rev Genet. 2008 Dec;9(12):938--950.

\bibitem{bib2}
Ohno S.
\newblock Evolution by gene duplication.
\newblock London: George Alien \& Unwin Ltd. Berlin, Heidelberg and New York:
  Springer-Verlag.; 1970.

\bibitem{bib3}
Magwire MM, Bayer F, Webster CL, Cao C, Jiggins FM.
\newblock {{S}uccessive increases in the resistance of {D}rosophila to viral
  infection through a transposon insertion followed by a {D}uplication}.
\newblock PLoS Genet. 2011 Oct;7(10):e1002337.

\end{thebibliography}

\end{document}

